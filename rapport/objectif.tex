\section{Objectifs}

Les besoins croissants en bande passante expliquent le recours
aux technologies Network On Chip. En effet, les réseaux sur puce offrent des
services plus étendus que les architectures à base de bus.


Dans le cadre général, la conception d'un réseau de communication commence
par la définition d'un modèle de communication. Dans notre cas, il s'agit de 32
processeurs qui cherchent à communiquer de manière indifférenciée à 4 mémoires
DDR2. Le sujet propose de diviser le réseau en deux sous-réseaux : une pour les
communications des processeurs vers les mémoires (requêtes) et un autre pour les
communication des mémoires vers les processeurs (réponses). Nous avons suivi
cette proposition et nous sommes intéressés au premier sous réseau.

Selon le cahier des charge, le réseau doit aussi :
\begin{itemize}
  \item garantir un débit de 100 Mbits/s pour chaque processeur,
  \item garantir une latence maximale de 80 ns point à point,
  \item permettre de configurer le réseau pour pouvoir utiliser différentes
    politiques d'arbitrage.
\end{itemize}


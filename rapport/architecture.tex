\section{Architecture - Modélisation SystemC}

Cette section décrit la modélisation du système en SystemC.

\subsection{Génération du traffic}

Les paquets se présentent sous la forme suivante:
\begin{itemize}
  \item 32 bits (\texttt{int data}) de donnée.
  \item 8 bits (\texttt{uint8\_t address}) d'adresse.
  \item Un champ libre (\texttt{mutable void* extra}) qui permet de rajouter de
    données de contexte suivant ce que l'on veut faire (tracing, statistics,
    \ldots). Celui-ci est rempli à différents endroits stratégiques
    (générateur de trafic, routeur) par des callbacks fournis aux instances
    concernées.
\end{itemize}

\vspace{0.5cm}

Les générateurs de trafic implémentés héritent tous d'une base commune \\
\texttt{noc::abstract\_traffic\_generator} qui leur permet :
\begin{itemize}
  \item D'émettre un paquet aléatoire : \texttt{void
    emit\_random\_package(void)}. Cette fonction remplit le champs \texttt{data}
    par une donnée aléatoire. En revanche entre deux appels, le champ
    \texttt{addresse} fournit reste la même pendant une certaine durée, qui elle
    est aléatoire. Ceci permet de simuler un trafic réaliste. \\
    Une fois le paquet créé, celui-ci est mis à disposition sur le champ
    \texttt{sc\_core::sc\_out<packet> output} et le champs
    \texttt{sc\_core::sc\_out<bool> activated} passe à \texttt{true}. Le champ
    \texttt{activated} reste à \texttt{true} et la fonction est bloquée tant que le champs
    \texttt{sc\_core::sc\_in<bool> acknoledge} n'est pas aquitté par le
    destinataire, ce qui permet d'être sûr de ne perdre aucun paquet sur la
    route. \\
    \textbf{Il faut donc toujours garder en tête que le générateur de traffic n'ira pas toujours
      à la vitesse demandée mais sera borné par la capacité à absorber les
      paquets du destinataire. Attention donc aux interprétations de résultat
    hâtives.}
  \item De pourvoir le module d'un callback de type
    \texttt{std::function<void(noc::packet)>} qui sera appelé à la création de
    chaque paquet.
\end{itemize}

\subsubsection{Les générateurs fournis}

\paragraph{\texttt{stream\_generator} \\}

Ce générateur de traffic émet des paquets à une fréquence fixe. On utilisera sa
methode \texttt{void set\_period(unsigned period)} pour régler cette fréquence.

\paragraph{\texttt{burst\_generator} \\}

Ce générateur possède trois paramètres:
\begin{itemize}
  \item La fréquence des bursts.
  \item La fréquence d'émission des paquets lors d'un burst.
  \item La quantité de paquet émits lors d'un burst.
\end{itemize}
On utilisera la methode \texttt{void set\_burst(unsigned long\_period, unsigned
short\_period, unsigned burst\_length)} pour fixer ces valeurs.

\subsection{Routeur}
Nous faisons correspondre à chaque entrée du système un module routeur, connecté
à autant de FIFO qu'il y a de sorties au système.

Chacun de ces modules reçoit alors les données correspondant à son entrée en
suivant un mécanisme de synchronisation de type handshake.

Sur un front montant de l'horloge, si le signal d'activation a une valeur positive,
le routeur lit un paquet et détermine la FIFO dans laquelle celui-ci doit être placé.
Le routeur lit ensuite le nombre d'emplacements libres de cette FIFO.
Si ce nombre est non nul, le paquet peut être écrit dans la FIFO.
Sinon, le routeur attend que la FIFO se désemplisse.
Une fois le paquet écrit dans la FIFO, le routeur émet un signal d'acquittement,
puis attend que le signal d'activation reprenne une valeur négative.
Une nouvelle valeur pourra alors être reçue suivant le même protocole.

\subsection{Arbitrage}
Similairement, nous faisons correspondre à chaque sortie du système un module
d'arbitrage, connecté à autant de FIFO qu'il y a d'entrées au système.
Chacun de ces modules reçoit également en entrée un signal définissant un choix
de stratégie d'arbitrage, déterminant dans quel ordre
les paquets en provenance des différentes entrées doivent être écrites en
sortie.


Dans la mesure où le choix de stratégie d'arbitrage peut changer à tout moment,
l'abitrage se fait en 5 temps:
\begin{enumerate}
\item lecture du choix de stratégie d'arbitrage
\item détermination de la FIFO sur laquelle lire le paquet (suivant la stratégie
d'arbitrage choisie)
\item lecture du paquet dans la FIFO correspondante (s'il y a au moins une FIFO
non vide)
\item écriture du paquet en sortie (ou d'une valeur par défaut si toutes les
FIFO sont vides)
\item et mise à jour de l'état interne du module en vue d'arbitrages futurs
\end{enumerate}
La suite de cette partie détaille les
différentes stratégies d'arbitrage modélisées et leur mise en oeuvre.

\subsubsection{Arbitrage suivant des priorités fixes}
Il s'agit de la stratégie d'arbitrage la plus simple: choisir la FIFO non vide la plus prioritaire.

Pour chaque FIFO, en suivant un ordre déterminé, le module lit donc le signal correspondant au nombre de paquets disponibles.
La première FIFO non vide est alors choisie.

\subsubsection{Arbitrage en tourniquet}
Il s'agit d'une stratégie assez similaires à la précédente.
Le module choisit ici aussi la première FIFO non vide, en parcourant celles-ci dans un ordre déterminé,
mais cette fois il les parcourt à partir de la dernière FIFO lue.

Cette stratégie d'arbitrage nécessite de savoir à tout instant quelle est la dernière FIFO à avoir été lue.
Il faut donc stocker cette information et la mettre à jour après chaque lecture.

\subsubsection{Arbitrage par choix de la moins récente utilisation}
Il s'agit cette fois de choisir la FIFO la moins récemment choisie.

La mise en oeuvre de cette stratégie repose sur l'utilisation d'un registre interne
dans lequel on stocke les indices des FIFOS par date de dernière utilisation, de la moins récente à la plus récente.
Le module  choisit alors la première FIFO non vide, en parcourant celles-ci dans l'ordre où leurs indices sont stockés dans le registre.

Après une lecture, pour mettre à jour le registre, on décale vers le début de celui-ci toutes les valeurs situées après celle correspondant à l'indice de la FIFO lue.
Cette dernière est ensuite écrite dans la dernière case du registre.

\subsubsection{Arbitrage de type FIFO}
Cette stratégie d'arbitrage est la plus complexe de celles présentées ici:
les paquets doivent être lus dans leur ordre d'arrivée, quelle que soit
la FIFO dans laquelle ils ont été placés.

Pour arriver à ce résultat, le module utilise un registre interne, initialement vide.
A chaque fois qu'une variable est écrite dans une des FIFOs, l'indice de cette FIFO est écrit dans ce registre
à une position correspondant au nombre de paquets présents dans l'ensemble des FIFOs.
Inversement, à chaque fois qu'un paquet est lu dans une FIFO, la première occurence de l'indice de cette
variable dans le registre est supprimée. Les valeurs du registres situées après cette occurence sont ensuite décalées vers le début du registre.

Ainsi, la première case du registre correspond toujours à l'indice de la FIFO contenant le paquet le plus ancien.

\subsubsection{Arbitrage aléatoire}
Pour choisir aléatoirement une FIFO dans laquelle lire, le module génère des nombres aléatoires
à l'aide d'un registre à décalage à rétroaction linéaire.
Ce nombre est ensuite ramené entre 0 et le nombre de FIFOs non vides, ce qui indique laquelle d'entre elles doit être choisie.


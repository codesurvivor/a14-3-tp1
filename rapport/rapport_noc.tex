\documentclass[11pt]{article}

\usepackage[utf8]{inputenc}
\usepackage[T1]{fontenc}
\usepackage[francais]{babel}
\usepackage[top=3cm, bottom=3cm, left=2.5cm, right=2.5cm]{geometry}
\geometry{a4paper} 

\usepackage{booktabs} 
\usepackage{array} 
\usepackage{multirow}
\usepackage{paralist} 
\usepackage{graphicx}
\usepackage{tabularx}
\usepackage{verbatim} 

\usepackage[output-decimal-marker={,}]{siunitx}
\usepackage{amsmath}
\usepackage{caption}
\usepackage{subcaption}
\usepackage{url}
\graphicspath{{figures/}}

\newcommand\cplot{\texttt{plot}}
\newcommand\cstairs{\texttt{stairs}}
\newcommand\SIi{\SI[separate-uncertainty]}
\renewcommand\tt\texttt

\title{Design d'un network on chip}

\begin{document}
\maketitle

\section{Objectifs}
	cadre noc
	architecture proposée
	contraintes perfs

\section{Architecture - Modélisation SystemC}
	Arbitreur
	Générateur de traffic
	Routeur
	Simulation (à déterminer)
	Protocole...

\section{Implémentation}

Cette section décrit un premier jet d'implémentation visant à déterminer un ordre de grandeur des performances atteignables et de la complexité du système

\subsection{Protocole}
La communication entre les composants se fait par un protocole à deux phases. Deux fils (requête et acquittement) sont ajoutés au bus de donné. Si ils sont au même niveau logique, le bus est considéré comme vide, sinon, il y a une donnée dessus. Ce protocole présente le désavantage de ne pas permettre de transmettre plus d'une donnée tous les deux cycles. En revanche, la conception de la FIFO est simplifiée.	

\subsection{FIFO}
Un étage de la FIFO est composé d'un banc de registre contenant au plus une donnée. On ajoute deux bascules contrôlant respectivement le fil de requête de cet étage et le fil d'acquittement du fil de l'étage précédent. Sur un front montant i l'étage précédent est plein et le présent est vide, le banc de registre copie la donnée et change l'état des bascules de contrôle. L'étage précédent est maintenant vide et l'étage actuel est plein.
Une FIFO est composée d'une succession, éventuellement vide, d'étages.

\subsection{Arbitre}
On utilise un codage one-hot pour désigner l'indice de la FIFO choisie. l'indice $-1$ est codé par "$0\dots0$"
\paragraph{Aléatoire}
L'implémentation reprend le code SystemC. L'indice de la FIFO choisie est extrait sous forme binaire et converti. L'indice choisi est renouvellé à chaque coup d'horloge.
\paragraph{Priorité fixe}
Un indice est choisi si et seulement si la FIFO correspondant n'est pas vide et aucun des indice plus prioritaire n'est choisi.
\paragraph{LRU}
les indices sont stockés du moins récemment utilisé au plus récemment utilisé. L'indice en tête est choisi. Pour la mise à jour, un indice progresse d'une case vers le moins récemment utilisé si l'indice réellement choisi a été moins récemment utilisé.
\paragraph{Priorité tournante}
L'arbitre stocke le dernier indice utilisé et accorde la priorité au suivant. Un indice est choisi si et sulement si il n'est pas vide et il est prioritaire ou alors le précédent est prioritaire et vide ou alors le précesseur du précédent est prioritaire ou vide\ldots
\paragraph{FIFO} Non implémenté.


\subsection{Routeur}

Pour chaque donnée reçue, le routeur décode l'adresse et inverse le bit de requête de la sortie correspondante. Les bits d'acquittements sont simplement aggrégé par des porte XOR et les données sont dirigées vers toutes les sorties.

\subsection{Performance et coût}

\subsubsection{Ressources logiques}
Le tableau \ref{perfs-sw-4} montre les ressources utilisées par un switch 4 vers 1 sur un FPGA Cyclone II. On remarque d'une part qu'en terme de ressources, les FIFOs sont rapidement les plus coûteuses (il y a quatre FIFO par switch et chaque étage a une largeur de $40+2$ bits

\begin{table}
\centering
\begin{tabular}{|c|c|c|}
\hline
Nombre d'étages de la fifo & Nombre de registres & Nombre de blocs logiques \\
\hline
0 & 93 & 181 \\
8 & 1405 & 258 \\
16 &  2717 & 312 \\
32 & 5341 & 452 \\
\hline \hline
éléments par étages & 164 & 8 \\
autres éléments & 93 & 184 \\
\hline
\end{tabular}
\caption{Ressources requises sur un Cyclone II par un switch 4 vers 1 selon la taille des FIFOs}
\label{perfs-sw-4}
\end{table}

\section{Méthodologie}

D'abord, il opportun de rappeler que ce projet a pour vocation de nous sensibiliser à la conception des MPOC pour une application donnée et dans un contexte donné. Mème si on a été amenés à concevoir une architecture symètrique, l'un des objectifs de ce module est de se démarquer de cette vision esthétique des systèmes on chip qui pousse à envisager des matrices régulières de communication et donc des architectures régulières.  

\subsection{Contraintes}

La conception se réalise par rapport à un espace de contraintes <Q,R,T>. 

Q designe les contraintes de qualité de service. Le modèle de communication à satisfaire est une communication indifférenciée des processeurs avec les mémoires avec des contraintes de débit et de latence.

R désigne les ressources disponibles. Dans notre cas, nous sommes 4 et disposons d'un certain nombre de machines. Au niveau des ressources logicielles, nous disposons de nombreuses copies de SystemC et une machine pour ISE (liscence d'évaluation 30 jours) et une autre pour Quartus et la chaîne Alliance.

T désigne le temps le temps de travail disponible. Nous avons totalisé environs 46 heures de travail jusqu'ici. L'utilisation de la méthodologie proposée nécessiterait probablement encore 50 heures.

\subsection{Méthodologie}

La méthodologie proposée se décompose en 3 étapes détaillées ci-dessous :
\begin{enumerate}
\item Le choix de l'espace de conception
\end{enumerate}

Dans la méthodologie de conception qu'on propose il y a 3 étapes : 
\paragraph{Choix de l'espace de conception}
Le but de cet étape est de définir les constantes et les variables qui seront utilisées dans l'exploration. Pour les constantes de conception, nous avons fixé un protocole, l'architecture des switchs et du routeur et une modèle. Les variables sont alors la taille des FIFOs et l'agencement des switchs.

\paragraph{Affectation des variables}

\paragraph{Extraction des caractéristiques de l'implémentation}
Une fois les paramètres d'architecture fixés, il définir un simulateur et une implémentation. L'implémentation doit fournir au simulateur les caractéristiques du réseau en fonction du des variables de conception. Cela peut passer par un générateur de code VHDL écrit par exemple en Perl et ensuite des scripts qui compilent ce code et extraient les contraintes d'horloge, les latences et les débits des switchs.

\paragraph{Simulation haut niveau et validation}
Le simulateur récupère ces informations et simule le système complet soumis au(x) modèle(s) de traffic pour obtenir des indicateurs globaux de performance. Si ceux-ci sont satisfaisant, on s'arrête. Sinon, on passe à l'étape suivante.

	

    2)Extraction des paramètres des éléments matéreils

    3)Simulation haut niveau et validation
	-Les paramètres récupérés des éléments matériels permettrons d'annoter les modèles haut niveau (SystemC)
	- La specification de l'application permet de définir un modèle de communication pour la simulation.
	-La simulation (atteinte rapidement) permettra de valider le comportement de chaque modèle. 
	

    Ouverture : adaptation au traffic réel
	<Q,R,T>
	Paramètres de choix : espace de conception
	Extraction des paramètres des éléments matéreils
	Simulation haut niveau et validation
	Ouverture : adaptation au traffic réel

\section{Réflexion}
	Conception
	Pareto
	Outils de CAO
	Simulation multiphysique
	Mur de la simulation et extraction de modèle
		équilibre on veut tout prendre en compte, on ne peut pas tout simuler en même temps
		des paramètres globaux dépendent grandement de l'implémentation (ex: conso)
	Repousser le front de Pareto ?
		Ouvrir des nouveaux paramètres (ex 3D : ajouter une dimension)
		Elargir les panels existant (ex : routeur optique)
		=> comment ? "Recherche" puis intégration
	Itération et convergence
		itération : architecture -> simule -> analyse -> nouvelle archi *améliorée*
		amélioration : analyse "intelligente" : idée de loi de variation, pb chaos non déterministe



\end{document}
